\documentclass{article}
\title{CS533R Course Note}
\author{Peiyuan Zhu}
\begin{document}
	\maketitle
	\section{}
	\begin{enumerate}
		\item Sence to image (CG) vs image to scene (CV)
		\item Possible project ideas:
		\item Pytorch: Number, Channel, Heigher, Weight; different for other
		\item Permute to get better performance to plot image
		\item Well-known datasets:
		\item Pytorch parallelism: train and load in parallel
		\item nn.Sequential: concatenate all
		\item Pytorch criterion: same as loss
		\item Cross-entropy loss
		\item Softmax
		\item Logsumexp
		\item Exp normalize
		\item L1 more robust than L2 because gives enough probability to outliers
		\item Train validation test
		\item 
	\end{enumerate}
	\section{NN architecture}
	\begin{enumerate}
		\item Minibatch
		\item SGD
		\item Adam
		\item Auto-diff
		\item Jacobian
		\item Efficient computation
		\item Gradient vanishing
		\item Relu
		\item Normalize input and output
		\item Initialization
		\item Xavier initialization (2010)
		\item He initialization
		\item batch normalization
		\item Pytorch: evaluation mode
		\item Regularication
		\item Dropout
		\item Weight decay 
		\item Prior
		\item Self-normalizing (2017)
		\item ResNet (2015): default
		\item DenseNet (2016): higher memory demand, slower
		\item Randomly wired network
		\item UNet: Conv+Max pool; one way: spatial dimension compressed, more channels; another way spatial dimension; network receives low level and higher level information
		\item Spatial down vs up sampling
		\item 
	\end{enumerate}
	\section{NN theory}
	\begin{enumerate}
		\item 
	\end{enumerate}
	\section{2D representation}
	\begin{enumerate}
		\item 
	\end{enumerate}
	\section{3D representation}
	\begin{enumerate}
		\item 
	\end{enumerate}
	\section{Shapes}
	\begin{enumerate}
		\item Voxel representation
		\item Surface mesh
		\item Graph convolution
		\item Spiral convolution
		\item Mesh Laplacian
		\item 
	\end{enumerate}
	\section{Graph Lapacian}
	\begin{enumerate}
		\item 
		\item 
	\end{enumerate}
	\section{Convolution}
	\begin{enumerate}
		\item 
		\item 
	\end{enumerate}
\end{document}
